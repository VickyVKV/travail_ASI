\documentclass[a4paper,11pt]{article} 

\usepackage[utf8]{inputenc} % encodé en utf-8
\usepackage[T1]{fontenc} % compatible avec les accents
\usepackage[top=3cm,bottom=4cm]{geometry} % gère les marges
\usepackage{graphicx} % gestion des images
\usepackage{array} % gestion des tableaux
\usepackage{csquotes} % gestion des guillemets
\usepackage{fourier} % utilise une autre police que celle par défaut (Computer Modern)
\usepackage[T1]{fontenc} 
\usepackage[
backend=biber,
style=numeric,
sorting=none
]{biblatex}
\addbibresource{bibliographie.bib}

\title{Web3 : Blockchain et NFT dans l'art}
\author{Vankerckhoven Vicky}
\date{STIC-B415 ULB 2022-2023}

\begin{document}

\maketitle


\tableofcontents
\break



\section{Introduction} % section 1
Cette étude a pour objectif d'identifier ce que le Web3 apporte dans l'art ainsi que son évolution. Je vais principalement me concentrer sur les NFT (Non Fungible Tokens) et la Blochain dans cet article.

% Explication Web3
La notion de Web3 est associée au web sémantique et est considérée comme un futur web décentralisé. En effet, la création du Web3 consiste à la participation et la coopération de ses utilisateurs leurs permettant de gérer leurs propres données. 

% Lien avec le sujet
Nous allons voir dans cette étude que le Web3 et en particulier les NFT et la blockchain ont leur part dans l'art, pouvant créer scandale ou apporter des évolutions notoires.

C'est un sujet qui est, aujourd'hui, en constante évolution, les informations à ce sujet sont donc variables. Cette étude va cependant essayer de donner une vue actuelle sur son impact dans l'art. 
\section{Définitions et principes} % section 2
\subsection{Web 3} % sous section 
Du web 1.0 au web 2.0 à la prochaine génération le web 3 ?

Le web 1.0 était le web statique qui est apparu dans les années 1990. Il était basé sur le partage d'informations avant tout. Du fait que c'est un web statique, aucune interaction n'était possible et, à cette époque, il y avait peu de design. 

Arrive assez rapidement le web 2.0, celui-ci permet aux utilisateurs d'interagir avec les sites et même de créer du contenu. Nous sommes donc passé dans du contenu dynamique. C'est le web que nous utilisons aujourd'hui aussi appelé le "web social". Il a permis la création des réseaux sociaux, des blogs ainsi que des sites de vidéos etc. Il est donc plus ou moins aisé d'utilisation et surtout accessible à tous. 

Le Web 3 est un nouveau concept émergeant, il s'agirait d'un web décentralisé étant géré par les utilisateurs entre eux et non plus par une seule entité centrale. Il s'agit donc d'une évolution du web comprenant diverses technologies : le web sémantique, l'intelligence artificielle, la réalité augmentée mais aussi la blockchain. 

\subsection{Blockchain} % sous section 
La blockchain a vu le jour suite à la crise économique de 2008, effectivement le bitcoin a été introduit comme nouvelle monnaie et ainsi un nouveau système de gestion monétaire fut créé. 

Le terme Blockchain signifie littéralement une chaîne de blocs. Dans cette chaîne, chaque bloc a un identifiant unique et contient une signature du bloc précédent. Une blockchain constitue une base de données sans autorité centrale et contenant donc un historique des données qui est infalsifiable. 
Infalsifiable car si un bloc est modifié ou une partie supprimée alors toute la chaîne perd sa cohérence, il est seulement possible d'ajouter des informations dans le bloc. 

Les blockchains sont basées sur un système peer-to-peer : c'est-à-dire qu'elles sont partagées sur des serveurs de plusieurs utilisateurs. Ces données sont mises à jour continuellement dans le monde entier. Cette caractéristique est essentielle, mais il y a donc un algorithme de consensus à respecter. Celui-ci va permettre un accord qui permet la cohérence de la blockchain.

\subsection{Non Fungible Tokens} % sous section 
Le terme "NFT", Non Fungible Tokens, a été introduit en 2017. Traduit en français il s'agit de jetons non fongibles, celui-ci est un ensemble de données numériques authentiques et non interchangeables sur une blockchain.  

Il s'agit en général d'illustrations ou d'oeuvres numériques possédant un certificat prouvant leur authenticité. Plusieurs projets ont été créés avant que le nom NFT leurs soient réellement associé, tel que "Rarepepes" présentant des cartes de Pepe the frog.  

Il est intéressant de noter que les NFT font partie de la technologie de Blockchain. Contrairement aux cryptomonnaies, les jetons non fongibles ont une valeur lié à un actif.  

\section{Le marché de l'art} % section 3
Aujourd'hui, le marché de l'art est réservé à une clientèle riche. Mais cela n'a pas toujours été le cas, fut un temps où l'art n'était pas vu de la même manière. C'est un secteur qui est en constante évolution. 
\subsection{Art digital} % sous section 
Il y a une transformation dans le domaine de l'art. Il apparait de plus en plus de galeries d'art en ligne ainsi que de plateformes d'art. Ce phénomène a explosé suite au COVID-19, obligeant les organisations à procéder autrement qu'en galeries, celles-ci étant fermées. Cette transformation digitale oblige les organisations à investir différement et revoir leurs stratégies. Effectivement, la communication se faisait avec un acheteur à la fois alors qu'aujourd'hui, avec les platerformes en ligne, la communication se fait avec plusieurs acheteurs potentiels simultanément. 

Les artistes eux-mêmes utilisent les réseaux sociaux et sites web pour vendre et mettre en valeur leurs arts. Ils passent donc de moins en moins via des intermédiaires. 
\section{Les opportunités dans l'art} % section 4

\section{Risques associés} % section 5
\subsection{Environnement} % sous section

\subsection{GDPR} % sous section
\section{Avenir ?} % section 6

\section{Conclusion} % section 7
\enquote{Une bonne conclusion est une conclusion finale.}

\break
\printbibliography[
heading=bibintoc,
title={Bibliographie}
]
\end{document} % fin du corps du texte